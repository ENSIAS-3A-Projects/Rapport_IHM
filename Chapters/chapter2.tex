\chapter{Phase de Définition (Analyser)}

\section{Nos Personas (Les Archétypes Validés)}

Pour incarner les besoins identifiés lors de la recherche, nous avons défini quatre profils utilisateurs : le cœur de cible (l'étudiante), le cas de rupture (l'automobiliste), l'opérationnel (le conducteur) et l'utilisateur externe (le touriste).

% --- PERSONA 1 : L'ÉTUDIANTE ---
\subsection{Persona 1 : Safae, "L'Étudiante Pressée"}
\textit{Le profil régulier (Daily Commuter) - 21 ans.}

\begin{figure}[H]
    \centering
    % Insérez ici l'image de la fiche ou la photo
    \includegraphics[width=1.1\linewidth]{imgs/safae.png}
\end{figure}

\begin{itemize}
    \item \textbf{Bio \& Contexte :} Safae prend le bus tous les jours à 7h30 pour aller à l'ENSIAS. "Digital Native", elle gère tout depuis son smartphone. Elle ne tolère pas l'attente injustifiée et oublie souvent son portefeuille (mais jamais son téléphone).
    \item \textbf{Buts (Goals) :}
    \begin{itemize}
        \item Connaître l'heure d'arrivée exacte du bus (Real-time).
        \item Gérer son abonnement mensuel directement sur l'appli.
        \item Utiliser son téléphone comme titre de transport.
    \end{itemize}
    \item \textbf{Frustrations (Pains) :}
    \begin{itemize}
        \item Le stress du "Bus fantôme" (pas d'info).
        \item Les files d'attente pour recharger sa carte physique.
        \item La peur d'être en retard en cours.
    \end{itemize}
\end{itemize}
\clearpage
% --- PERSONA 2 : L'AUTOMOBILISTE ---
\subsection{Persona 2 : Sarah, "L'Automobiliste Anxieuse"}
\textit{Le cas de rupture (Utilisateur occasionnel) - 31 ans.}

\begin{figure}[H]
    \centering
    \includegraphics[width=1.1\linewidth]{imgs/automobiliste.png}
\end{figure}

\begin{itemize}
    \item \textbf{Bio \& Contexte :} Sarah ne prend jamais le bus. Suite à une panne de voiture, elle doit l'utiliser en urgence. Elle n'a pas d'espèces ("Cashless") et ne connaît pas les lignes. Elle est en situation de stress et d'anxiété sociale.
    \item \textbf{Buts (Goals) :}
    \begin{itemize}
        \item Être guidée pas-à-pas comme avec un GPS.
        \item Payer sans contact (Apple Pay/Carte) pour éviter la gestion de la monnaie.
        \item Se rassurer en voyant le bus avancer sur la carte.
    \end{itemize}
    \item \textbf{Frustrations (Pains) :}
    \begin{itemize}
        \item Ne pas avoir de monnaie pour payer le chauffeur.
        \item La peur de se tromper de direction ou de rater l'arrêt.
        \item Le sentiment d'être une "touriste" dans sa propre ville.
    \end{itemize}
\end{itemize}
\clearpage
% --- PERSONA 3 : LE CONDUCTEUR ---
\subsection{Persona 3 : Ayoub, "Le Conducteur / Staff"}
\textit{L'Admin terrain et garant de l'efficacité - 31 ans.}

\begin{figure}[H]
    \centering
    \includegraphics[width=1.1\linewidth]{imgs/chauffeur.png}
\end{figure}

\begin{itemize}
    \item \textbf{Bio \& Contexte :} Chauffeur expérimenté, Ayoub aime conduire mais est usé par la gestion de la caisse à bord qui crée des retards. Il veut se concentrer sur la route, pas sur la comptabilité.
    \item \textbf{Buts (Goals) :}
    \begin{itemize}
        \item Fluidifier la montée des passagers (Validation autonome).
        \item Recevoir son itinéraire et les alertes sur sa tablette.
        \item Signaler un incident en un clic (Gros bouton).
    \end{itemize}
    \item \textbf{Frustrations (Pains) :}
    \begin{itemize}
        \item Les disputes liées au rendu de monnaie.
        \item Les appels téléphoniques de la régulation pendant la conduite.
        \item Ne pas savoir s'il est en retard sur son horaire.
    \end{itemize}
\end{itemize}
\clearpage
% --- PERSONA 4 : LE TOURISTE ---
\subsection{Persona 4 : Manuel, "Le Touriste Explorateur"}
\textit{Le besoin de simplicité et de paiement international - 39 ans.}

\begin{figure}[H]
    \centering
    \includegraphics[width=1.1\linewidth]{imgs/touriste_fiche.png}
\end{figure}

\begin{itemize}
    \item \textbf{Bio \& Contexte :} Visiteur étranger présent pour 3 jours. Il ne parle pas la langue et ne veut pas retirer de cash. Il veut visiter les monuments ("Tour Hassan") sans se soucier de la complexité du réseau.
    \item \textbf{Buts (Goals) :}
    \begin{itemize}
        \item Payer avec sa carte internationale in-app.
        \item Sélectionner des points d'intérêt touristiques sur la carte.
        \item Être géolocalisé pour ne pas se perdre.
    \end{itemize}
    \item \textbf{Frustrations (Pains) :}
    \begin{itemize}
        \item La barrière de la langue avec le chauffeur.
        \item La complexité des zones tarifaires.
        \item L'obligation d'avoir de la monnaie locale.
    \end{itemize}
\end{itemize}

\section{Carte d'Empathie (Empathy Map)}

Focus sur notre cœur de cible, \textbf{Safae (L'Étudiante)}, lors de l'attente à l'arrêt.

\begin{table}[H]
    \centering
    \begin{tabular}{|p{0.45\textwidth}|p{0.45\textwidth}|}
        \hline
        \textbf{Ce qu'elle VOIT (See)} \& \textbf{Ce qu'elle ENTEND (Hear)} \\
        \begin{itemize}
            \item Le bus partir au loin.
            \item Une foule qui attend.
            \item Des panneaux horaires vides.
        \end{itemize} \& 
        \begin{itemize}
            \item Les gens souffler d'impatience.
            \item Le bruit de la circulation.
            \item "Préparez la monnaie !"
        \end{itemize} \\
        \hline
        \textbf{Ce qu'elle RESSENT (Feel)} \& \textbf{Ce qu'elle FAIT (Do)} \\
        \begin{itemize}
            \item Frustration (retard).
            \item Stress (oubli de carte).
            \item Envie de transparence.
        \end{itemize} \& 
        \begin{itemize}
            \item Regarde son téléphone (compulsif).
            \item Cherche sa carte dans son sac.
            \item Se plaint sur WhatsApp.
        \end{itemize} \\
        \hline
    \end{tabular}
    \caption{Carte d'Empathie : Safae à l'arrêt}
\end{table}

\section{Définition du Problème (Problem Statement)}

L'analyse croisée de nos quatre personas met en lumière une fracture numérique critique. Le problème ne se limite pas au paiement ponctuel, mais concerne l'ensemble de l'expérience de mobilité.

Nous devons répondre simultanément aux besoins de l'usager régulier (Abonnements, Temps réel) et de l'usager occasionnel (Ticket unique, Guidage).

Nous formulons donc notre défi majeur ("How Might We") ainsi :

\begin{center}
    \fbox{\begin{minipage}{0.95\textwidth}
        \centering
        \vspace{0.4cm}
        \Large
        \textbf{\textit{"Comment pourrions-nous centraliser l'expérience de transport pour offrir une visibilité totale (Temps Réel) et une dématérialisation complète (Abonnements/Tickets) afin de supprimer le stress et le cash ?"}}
        \vspace{0.4cm}
    \end{minipage}}
\end{center}

Cette problématique se décompose en trois axes de conception pour la suite du projet :

\begin{enumerate}
    \item \textbf{L'Axe Informationnel (Réassurance) :} Comment garantir la géolocalisation précise et l'horaire en temps réel pour qu'un étudiant ou un touriste n'attende plus jamais "à l'aveugle" ?
    \item \textbf{L'Axe Transactionnel (Fluidité) :} Comment permettre l'achat instantané de tout titre de transport (du ticket unitaire à l'abonnement annuel) sans aucune interaction physique ni espèce ?
    \item \textbf{L'Axe Opérationnel (Efficacité) :} Comment réduire la charge mentale du conducteur en automatisant la validation et le guidage ?
\end{enumerate}