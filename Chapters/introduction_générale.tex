\chapter*{Introduction Générale}
\addcontentsline{toc}{chapter}{Introduction Générale}

L'urbanisation croissante des métropoles modernes place la mobilité au cœur des enjeux sociétaux. Si les infrastructures de transport (bus, tramways) constituent la colonne vertébrale de cette mobilité, l'expérience vécue par l'usager reste trop souvent synonyme de friction : incertitude quant aux horaires, complexité des systèmes de paiement et manque de visibilité globale sur le réseau. Dans ce contexte, la technologie ne doit pas seulement servir à optimiser les flux logistiques, mais avant tout à apaiser et fluidifier le parcours de l'humain.

Ce projet, intitulé \textbf{"UrbanMove"}, s'inscrit dans une démarche de \textbf{Conception Centrée Utilisateur (CCU)}. Notre ambition n'est pas uniquement de développer une solution logicielle fonctionnelle, mais de concevoir une \textbf{Interaction Homme-Machine (IHM)} capable de réduire la charge mentale des voyageurs et des opérateurs. Comment transformer l'attente anxieuse à un arrêt de bus en une expérience maîtrisée ? Comment rendre le paiement du transport aussi invisible que possible ?

Pour répondre à cette problématique, nous avons adopté une méthodologie structurée allant de l'analyse psychologique des besoins à la réalisation d'interfaces haute fidélité. Ce rapport retrace les étapes clés de notre réflexion ergonomique :

\begin{itemize}
    \item Le \textbf{Chapitre 1} dresse un état de l'art et analyse le contexte de la mobilité urbaine, mettant en lumière les limites des systèmes actuels.
    \item Le \textbf{Chapitre 2} se concentre sur la phase de définition. À travers l'élaboration de Personas et de cartes d'empathie, nous identifions les points de douleur spécifiques de nos différents profils d'utilisateurs.
    \item Le \textbf{Chapitre 3} détaille la phase de conception (Idéation). Nous y exposons nos choix de design, les parcours utilisateurs (User Flows) et les principes d'IHM retenus pour garantir l'utilisabilité.
    \item Enfin, le \textbf{Chapitre 4} présente la phase de prototypage. Il dévoile les maquettes finales de l'application voyageur et du tableau de bord administrateur, concrétisant notre vision d'une mobilité connectée et humaine.
\end{itemize}

À travers ce projet, nous démontrons que l'interface est bien plus qu'une couche visuelle : elle est le point de rencontre essentiel entre la performance technologique et la satisfaction de l'usager.